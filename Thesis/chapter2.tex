\chapter{مفاهیم}
\section{اینترنت اشیا}
اینترنت اشیا که از آن به عنوان "انقلاب صنعتی جدید" یاد می‌شود، به دلیل تغییری که در شیوه
زندگی، کار، سرگرمی و مسافرت مردم و ... ایجاد کرده، تعاملات بین دولت‌ها و دنیای پیرامون‌شان را با دنیای مجازی و تكنولوژی نیز دگرگون ساخته است. ورود دستگاه اتومبیل با مجموعه‌ای از نرم‌افزارهای کاربردی جهت ایجاد تعامل بین کاربر، خانه‌ها و ساختمان‌های هوشمند، امكان پخش موسیقی تنها با ادای چند کلمه و هزاران کاربرد دیگر در مدیریت هوشمند شهر، حمل و نقل، کشاورزی، صنایع دفاعی،
صنعت بیمه، صنایع مربوط به نفت، گاز و معدن، مدیریت انرژی، پایش و امنیت اماکن عمومی و
خصوصی، خرده فروشی، بانک‌ها، بهداشت و درمان، هتل‌داری، مهر تاییدی بر اهمیت اینترنت اشیا است.


اینترنت اشیا، برای نخستین بار در سال ۱۹۹۹ توسط کوین اشتون مورد استفاده قرار گرفت و جهانی را توصیف کرد که در آن هر چیزی، از جمله اشیا بی‌جان، برای خود هویت دیجیتال داشته باشند و به کامپیوترها اجازه دهند تا آن ها را سازماندهی و مدیریت کنند.


در سال‌های بعد، تعاریف دیگری از اینترنت اشیا توسط افراد و شرکت‌های مختلف ارائه گردید. اینترنت اشیا مفهومی جدید در دنیای فناوری و ارتباطات است که به طور خلاصه می‌توان گفت، اینترنت اشیا فناوری مدرنی است که در آن برای هر موجودی (انسان، حیوان و یا اشیا) قابلیت ارسال داده از طریق شبكه‌های ارتباطی، اعم از اینترنت یا اینترانت، فراهم می‌گردد. در این فناوری، اشیا پیرامون ما قادرند از محیط اطراف خود داده‌های مفیدی را از طریق حسگرهای مختلف جمع‌آوری کرده و آن‌ها را برای پردازش و اتخاذ تصمیمات لازم به یک سیستم مرکزی منتقل کنند. در واقع ایده کلی فناوری اینترنت اشیا دریافت، ذخیره‌سازی و ارسال اطلاعات از محیط به منظور تحلیل آن‌ها و در نهایت ارائه خدمات بهتر و هوشمندتر به کاربر نهایی است. به عبارتی اینترنت اشیا را می‌توان به عنوان تكامل بعدی اینترنت دانست که جهش بزرگی در توانایی جمع‌آوری، تحلیل و توزیع داده دارد.


اینترنت اشیا یک شبكه داخلی متشکل از دستگاه‌های فیزیکی، وسایل نقلیه، ساختمان‌ها، سایر موارد الكترونیكی، نرم‌افزارها، حسگرها، محرك‌ها و یک اتصال به شبكه که اشیا را به جمع‌آوری و تبادل
داده‌ها قادر می‌سازد، می‌باشد. در سال ۲۰۱۳ میلادی استانداردهای جهانی، برای اینترنت اشیا تعریف "زیرساخت‌های جامعه اطلاعاتی" را مطرح کردند. اینترنت اشیا امكان حس و کنترل شدن اشیا از راه دور را با استفاده از زیرساخت شبكه فراهم می‌سازد، فرصت ادغام مستقیم دنیای فیزیكی با سیستم‌های کامپیوتری را بالا می‌برد و در نتیجه بهبود بهره‌وری، دقت و سود اقتصادی را علاوه بر کاهش دخالت انسان به همراه دارد.
\section{سیر تکامل}
تغییراتی که در تعامل و ارتباطات بین موجودیت‌های (اشیای) موجود در جهان، در طول زمان (قبل از ظهور اینترنت، بعد از ظهور آن تا مطرح شدن ایده اینترنت اشیا) بوجود آمده، در شكل ۲-۱ به
تصویر کشیده شده است:


تكام در ارتباطات شامل مراحل زیر است:
\begin{enumerate}
	\item
	 قبل از اینترنت: ارتباط انسان با انسان
	\item
	اینترنت محتوا: وب نسل ۱ بر روی بستر شبکه‌های \lr{IP}
	\item 
	اینترنت سرویس: وب نسل ۲ و امکان تولید محتوا توسط کاربران
	\item 
	اینترنت افراد: شبکه‌ها و رسانه‌های اجتماعی
	\item
	اینترنت اشیا: اتصالات ماشین به ماشین
\end{enumerate}
ارتباطات ماشین به ماشین \RTLfootnote{\lr{M2M}} اصطلاحی است که برای توصیف هر فناوری که دستگاه‌های شبكه را قادر به تبادل اطلاعات و انجام برخی عملیات بدون دخالت انسان می‌کند، به کار گرفته می‌شود. در واقع به عنوان بخشی از اینترنت اشیا در نظر گرفته می‌شود. پیش‌بینی‌ها نشان می‌دهد که تعداد اتصالات اینترنت اشیا در سال‌های آینده رشد قابل توجهی خواهد داشت.


مفاهیم مرتبط به اینترنت اشیا سال‌ها قبل توسط مارك ویسر در شرکت زیراکس مطرح شده بود و
در قالب حوزه پردازش فراگیر در حال رشد بود. هدف حوزه پردازش فراگیر شكل‌گیری جهانی است که در آن اشیا اطراف ما (که به طور روزمره با آن‌ها سر و کار داریم) دارای قدرت پردازش بوده و به
صورت بی‌سیم یا کابلی با شبكه جهانی در ارتباط باشند. دورنمای دیدگاه اولیه مارك ویسر دستیابی به سیستم‌های شبكه‌ای در حوزه فناوری اطلاعات و ارتباطات است که نهفته در محیط پیرامون، پنهان از دید کاربر و خودکار هستند. چنین سیستم‌هایی کاربر و فعالیت‌هایش را دنبال کرده و به نیازمندی‌های آن پاسخ می‌دهد. لذا به عبارتی کاربر را قادر می‌سازد با محیط فیزیكی اطرافش سازگار شود و سرویس هوشمندتری را دریافت کند.

\section{اکوسیستم اینترنت اشیا}
\subsection{اکوسیستم}
با توجه به تعاملی که اینترنت اشیا بین عناصر و اشیا متنوع ایجاد کرده و با درنظر گرفتن این موضوع که فناوری‌های اینترنت اشیا مختص یک صنعت و یا زنجیره تامین خاصی نیستند، این مسئله باعث ایجاد روابط پیچیده‌ای می‌شود که شامل تعداد زیادی از بازیگران خواهد بود. در این صورت بایستی با استفاده از مفهوم اکوسیستم، به این شرایط پیچیده سامان بخشید. به عنوان مثال، اکوسیستم پیشنهادی شرکت \lr{IDC} برای اینترنت اشیا در شكل ۲-۲ قرار داده‌ شده‌ است:


هدف اینترنت اشیا قابلیت‌ بخشی به اشیا برای ارتباط با هم در همه جا، هر زمان و با هر وسیله،
راه و شبكه‌ ارتباطی است. با این وجود اینترنت اشیا، تعامل چند دنیا است که دارای اجزای مختلفی است که خود حاصل تعامل سه دنیا است و گاها از این سه دنیا تحت عنوان اجزای اینترنت اشیا نیز یاد می‌شود که عبارتند از:
\begin{enumerate}
	\item دنیای دیجیتال
	\item دنیای سایبری
	\item دنیای فیزیکی واقعی
\end{enumerate}
\section{سیستم ردیابی}
همانطور که گفتیم اینترنت اشیا چهارچوبی است که در آن هر شی دارای قابلیت اتصال به اینترنت است. \RTLfootnote{\href{http://www.wikicfp.com/cfp/servlet/event.showcfp?event
		id=40524&amp;copyownerid=6818}{\lr{ref}}}
	این اتصال به اینترنت دستگاه‌های محاسباتی تعبیه شده در هر شی، آن را قادر به ارسال و دریافت اطلاعات می‌کند. اینترنت اشیا به طور گسترده به گسترش ارتباط شبکه‌ای و قابلیت محاسبه اشیا، دستگاه‌ها، حسگرها و هر چیزی که به طور عادی کامپیوتر حساب نمی‌شود، اشاره دارد. کوین اشتون، پدر اینترنت اشیا، اشاره می‌کند که "اطلاعات و داده‌ها راه عالی برای کاهش هدرروی و افزایش بهره‌وری می‌باشد و این دقیقا آن چیزی است که اینترنت اشیا فراهم می‌کند." این اشیا هوشمند دارای حداقل نیاز انسان برای تولید، مبادله و محاسبه داده می‌باشند. زمینه‌های کاربرد تکنولوژی اینترنت اشیا روز به روز در حال افزایش است. یکی از کاربردهای اینترنت اشیا در ردیابی وسایل نقلیه است.\cite{Mangla2017, Mukhtar2015}
	
انتظار می‌رود تعداد کل وسایل نقلیه روز به روز افزایش یابد با توجه به اینکه مالکیت آن‌ها برای افراد با توحه به اقتصاد رو به رشد کشورهایی مثل چین و هند، مقرون به صرفه می‌باشد. اما با این وجود کمبود سیستم ردیابی هم‌چنان حس می‌شود. چنین سیستمی در زمینه‌های مختلفی از چمله امنیت خودرو شخصی، وسایل نقلیه عمومی، مدیریت ناوگان و ... کاربرد دارد. \cite{Pham2013} امروزه سیستم‌های ردیابی مختلفی در بازار وجود دارد اما آن‌ها دارای کاربرد مشخص، ناحیه کاری مشخص و عمدتا هم بسیار هزینه‌بر می‌باشند.\cite{YoujingCui2003} بنابراین سیستم ردیابی طراحی شده برای امنیت خودرو برای مدیریت ناوگان مناسب نمی‌باشد. \cite{Song2008} پس باید سیستمی را پیاده‌سازی کنیم که به راحتی برای کاربردهای مختلف قابل استفاده و مقرون به صرفه باشد.


همانطور که اشاره کردیم سیستم ردیابی راه‌حلی برای مدیریت ناوگان و امنیت اشیا متحرک از جمله افراد، وسایل نقلیه و ... می‌باشد. این تکنولوژی برای مشخص کردن مکان شی متحرک در هر زمان از متدهای مختلفی مثل سیستم موقعیت‌یاب جهانی \RTLfootnote{GPS} که به ماهواره‌های اطراف پایگاه زمین متصل می‌باشد، استفاده می‌کند. سیستم‌های ردیابی مدرن از تکنولوژی \lr{GPS} برای مانیتور و موقعیت‌یابی وسایل نقلیه در هر کجای زمین استفاده می‌کنند. این سیستم ردیابی در داخل خودرو یا هر شی متحرکی که می‌خواهیم آن را ردیابی کنیم، قرار می‌گیرد و امکان موقعیت‌یابی در هر لحظه را فراهم می‌آورد  و اطلاعات لازم برای ردیابی را در اختیار افراد می‌گذارد. این سیستم یک دستگاه ضروری برای ردیابی اشیا متحرک مخصوصا وسایل نقلیه است که صاحبان آن‌ها بتوانند هر زمان وسیله نقلیه خود را ردیابی کنند. هم‌چنین از این سیستم می‌توان برای ردیابی و پیدا کردن وسایل دزدیده شده استفاده نمود. داده جمع‌آوری شده از طریق این سیستم را می‌توان بر روی نقشه الکترونیکی از طریق اینترنت یا نرم‌افزارهای کاربردی مشاهده نمود.  


