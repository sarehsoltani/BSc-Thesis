\chapter{مفاهیم}
\section{اینترنت اشیا}
اینترنت اشیا که از آن به عنوان "انقلاب صنعتی جدید" یاد می‌شود، به دلیل تغییری که در شیوه
زندگی، کار، سرگرمی و مسافرت مردم و ... ایجاد کرده، تعاملات بین دولت‌ها و دنیای پیرامون‌شان را با دنیای مجازی و تكنولوژی نیز دگرگون ساخته است. ورود دستگاه اتومبیل با مجموعه‌ای از نرم‌افزارهای کاربردی جهت ایجاد تعامل بین کاربر، خانه‌ها و ساختمان‌های هوشمند، امكان پخش موسیقی تنها با ادای چند کلمه و هزاران کاربرد دیگر در مدیریت هوشمند شهر، حمل و نقل، کشاورزی، صنایع دفاعی،
صنعت بیمه، صنایع مربوط به نفت، گاز و معدن، مدیریت انرژی، پایش و امنیت اماکن عمومی و
خصوصی، خرده فروشی، بانک‌ها، بهداشت و درمان، هتل‌داری، مهر تاییدی بر اهمیت اینترنت اشیا است.


اینترنت اشیا، برای نخستین بار در سال ۱۹۹۹ توسط کوین اشتون مورد استفاده قرار گرفت و جهانی را توصیف کرد که در آن هر چیزی، از جمله اشیا بی‌جان، برای خود هویت دیجیتال داشته باشند و به کامپیوترها اجازه دهند تا آن ها را سازماندهی و مدیریت کنند.


در سال‌های بعد، تعاریف دیگری از اینترنت اشیا توسط افراد و شرکت‌های مختلف ارائه گردید. اینترنت اشیا مفهومی جدید در دنیای فناوری و ارتباطات است که به طور خلاصه می‌توان گفت، اینترنت اشیا فناوری مدرنی است که در آن برای هر موجودی (انسان، حیوان و یا اشیا) قابلیت ارسال داده از طریق شبكه‌های ارتباطی، اعم از اینترنت یا اینترانت، فراهم می‌گردد. در این فناوری، اشیا پیرامون ما قادرند از محیط اطراف خود داده‌های مفیدی را از طریق حسگرهای مختلف جمع‌آوری کرده و آن‌ها را برای پردازش و اتخاذ تصمیمات لازم به یک سیستم مرکزی منتقل کنند. در واقع ایده کلی فناوری اینترنت اشیا دریافت، ذخیره‌سازی و ارسال اطلاعات از محیط به منظور تحلیل آن‌ها و در نهایت ارائه خدمات بهتر و هوشمندتر به کاربر نهایی است. به عبارتی اینترنت اشیا را می‌توان به عنوان تكامل بعدی اینترنت دانست که جهش بزرگی در توانایی جمع‌آوری، تحلیل و توزیع داده دارد.