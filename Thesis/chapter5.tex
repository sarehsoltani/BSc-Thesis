\chapter{جمع‌بندي و کارهای آتی}
\section{جمع‌بندی}
در این پروژه هدف ما پیاده‌سازی یک سیستم ردیابی بی‌درنگ برای ردیابی اشیا متحرک بوده است.  لذا در این پروژه سیستم ردیابی، برای مانیتور کردن موقعیت شی متحرک از طریق پیام کوتاه و همچنین به صورت آنلاین بر روی نقشه تست و پیاده‌سازی نمودیم. هسته اصلی سیستم طراحی شده، برد آردوینو و ماژول سیم ۸۰۸ می‌باشد. مودم جی‌اس‌ام با استفاده از دستورات \lr{AT} کنترل شده و از این طریق امکان تبادل اطلاعات با استفاده از شبکه جی‌اس‌ام را فراهم می‌آورد. در این پروژه برای یافتن موقعیت شی از ماژول جی‌پی‌اس استفاده شده است. جی‌پی‌اس هر دو دقیقه یکبار موقعیت مکانی شی را از ماهواره دریافت کرده و این اطلاعات به سمت سرور فرستاده می‌شوند. سرور اطلاعات دریافتی را برای وب سرویس پیاده‌سازی شده، ارسال می‌کند. در نهایت این داده‌ها در پایگاه داده ذخیره شده و در برنامه کاربردی نمایش داده می‌شوند و میتوانیم مسیر حرکت شی را بر روی نقشه مشاهده کنیم.


برنامه پیاده‌سازی شده بر روی آردوینو به این صورت نوشته شده است که هر دو دقیقه یکبار موقعیت مکانی شی را با استفاده از جی‌پی‌اس دریافت می‌کند و به سرور می‌فرستد. در سمت سرور  این اطلاعات در پایگاه داده مانگو ذخیره شده و سپس با استفاده از این اطلاعات مسیر حرکت شی بر روی نقشه داده می‌شود. کد سمت سرور به این صورت نوشته شده است که هر یک دقیقه یکبار به پایگاه داده مراجعه کرده و نقشه حاوی مسیر حرکت شی به‌روزرسانی می‌شود.


در پایان این پروژه توانستیم سیستم ردیابی بی‌درنگی را  طراحی کنیم که امروزه دارای کاربرد فراوان در زمینه‌های مختلف مانند ردیابی وسایل نقلیه، کودکان و سالمندان و ... می‌باشد و با استفاده از آن قادر خواهیم بود اقدامات لازم را در سریع‌ترین زمان ممکن انجام دهیم.