\chapter{مقدمه}

\section{اینترنت اشیا}
اینترنت اشیا \RTLfootnote{\lr{Internet of Things}} مجموعه وسيعي از سنسورها و عملگرهايي است كه شرايط مختلف محیط را اندازه‌گیری و پردازش مي‌كنند. در سال‌هاي اخير فناوري اينترنت اشيا رشد چشمگيري داشته و توانسته در زمينه‌هاي مختلف، نيازهاي متعدد و پيچيده‌اي را برطرف كند.
\section{معرفی سیستم ردیابی}
اینترنت اشیاء یک بستر ارتباطی جدید در جهت برقراری ارتباط بین اشیا هوشمند می‌باشد. معرفی این بستر موجب شــده اســت تا امکانات جدیدی برای حل مسائلی همچون تعیین مکان و ردیابی اشیا متحرک از حمله وسایل نقلیه در سطح یک شهر، منطقه یا کشور فراهم گردد.
اینترنت اشیا یک بستر ارتباطی جدید است که به سرعت در حال بدست آوردن راهکارهایی در رابطه با سناریوی ارتباط از راه دور می‌باشد و انتظار می‌رود که مبادله اطلاعات در رابطه با هر شی در شبکه‌های زنجیره‌ای منابع جهانی را آسان کند، شفافیت را افزایش دهد و کارایی‌شان را بالا ببرد. به طور گسترده اینترنت اشیا می‌تواند به عنوان ستون اصلی سیستم‌های فراگیر و فعال‌سازی محیط‌های هوشمند برای سادگی در تشخیص و شناسایی اشیا و بازیابی اطلاعات از اینترنت در هر زمان و در هر مکان به کار برده شود.


از یک دیدگاه مفهومی، اینترنت اشیا متکی بر سه اصل مرتبط با توانایی اشیا هوشمند است: ۱- قابلیت شناسایی (هر چیزی خود را شناسایی کند) ۲- قابلیت انتقال (هر چیزی دست به انتقال می‌زند) ۳- قابلیت تعامل (هر چیزی دست به تعامل می‌زند) یا در میان خودشان و یا با کاربران نهایی یا سایر نهادهای فعال در شبکه. اشیا معمولا یا به صورت منحصر به فردو یا به عنوان عضوی از یک رده شناسایی می‌شوند.


یکی از مساپل مطرح امروزی، ردیابی بی‌درنگ اشیا متحرک می‌باشد که به ردیابی بی‌درنگ موقعیت فعلی یک شی متحرک معین اشاره دارد. 
سیستم ردیابی اشیا متحرک یک راه‌حل برای بسیاری از مشکلات از جمله مساپل امنیتی است. تکنولوژی است که برای مشخص کردن موقعیت شی مورد استفاده قرار می‌گیرد.
   
\section{این همه فایل؟!}\label{sec2}

\section{از کجا شروع کنم ساره؟}

\section{مطالب پایان‌نامه را چطور بنویسم؟}
\subsection{نوشتن فصل‌ها}

\subsection{مراجع}

\subsection{واژه‌نامه فارسی به انگلیسی و برعکس}

\section{اگر سوالی داشتم، از کی بپرسم؟}