\chapter{مقدمه}
مفهوم اينترنت اشيا \RTLfootnote{\lr{Internet of Things (IOT)}} به اشيايي با هويت خاص اطلاق مي‌شود كه دارای شناسه منحصر به فرد بوده و توانايي انتقال داده روی شبکه، بدون نياز به تعامل و دخالت انسان را دارند. هدف اصلي آن هوشمند سازی اشيا و فراهم آوردن بستری است كه از طريق آن اشيا قادر به ارسال و دريافت اطلاعات با يکديگر مي‌باشند. اينترنت اشيا به طور گسترده به توسعه قابليت محاسبه و ارتباطات شبکه‌ای اشيا، دستگاه‌ها، سنسورها يا هر مورد ديگری كه به طور معمول به عنوان كامپيوتر در نظر گرفته نمي‌شود، اشاره دارد. اين اشيای هوشمند دارای قابليت جمع‌آوری داده از راه دور، تحليل و مديريت آن‌ها هستند. \cite{Mukhtar2015}

اینترنت اشیا مجموعه وسيعي از سنسورها و عملگرهايي است كه شرايط مختلف محیط را اندازه‌گیری و پردازش مي‌كنند. در سال‌هاي اخير فناوري اينترنت اشيا رشد چشمگيري داشته و توانسته در زمينه‌هاي مختلف، نيازهاي متعدد و پيچيده‌اي را برطرف كند. به علت گسترش فناوري‌هاي جديد، توليد سنسورهاي هوشمند، رشد تكنولوژي‌هاي ارتباطي و پيچيده شدن نيازها، اينترنت اشيا قدرت زيادي پيدا كرده و در زمینه‌های مختلف از آن استفاده مي‌شود و باعث گسترش سيستم‌هاي هوشمند در محیط شده است. این سیستم‌ها برای اینکه بتوانند اثر مثبتی بر محیط بگذارند باید با یکدیگر در تعامل باشند. فناوري‌هاي مبتني بر اينترنت اشيا نيازمندي‌هاي متفاوتي در مقايسه با ساير فناوري‌ها دارند. به طور معمول اين سيستم‌ها حافظه، توان مصرفي و پهناي باند كمتري نسبت به ساير سيستم‌ها دارند. اكثر سيستم‌هاي هوشمند مبتني بر باتري هستند و در مكاني دوردست قرار دارند به گونه‌ای كه نمیتوان به صورت مداوم آن‌ها را شارژ كرد. در نتيجه توان مصرفي و محدوده قابل پوشش براي اين سيستم‌ها به ويژه آن‌هايي كه در سطح كلان اجرا مي‌شوند مانند كشاورزي هوشمند، شهر و خانه هوشمند و مسائل رديابي، مسئله بسيار مهمي است. پروتكل‌هاي ارتباطي بي‌سيم متعددي وجود دارد كه هر كدام ويژگی منحصر به فرد خود را دارند.

يكي از كاربردهاي مهم اينترنت اشيا، سامانه‌هاي رديابي است كه در فناوري‌هاي مختلف مورد استفاده قرار مي‌گیرد. سيستم‌هاي رديابي براي اولين بار براي صنعت حمل و نقل به وجود آمدند. از نيازهاي اساسي صاحبان اين صنعت، بررسي موقعيت وسايل نقليه است. ابتدايي‌ترين سيسبتم‌هاي ساخته شده براي يافتن موقعيت، سيسبتم‌هاي غير فعال بودند كه اطلاعات را در حافظه‌اي ذخیره مي‌كردند و دسترسي به آن‌ها تنها زماني ممكن بود كه وسيله نقليه در دسترس باشد. اين نوع سيستم‌ها براي كاربردهاي بلادرنگ مناسب نيستند چون در اين كاربردها نياز است اطلاعات بلافاصله در اختيار كاربر قرار بگيرد. براي برطرف كردن اين نياز، سيستم‌هاي فعال به وجود آمدند كه با استفاده از يک سخت‌افزار تعبيه شده در وسيله نقليه و سرور رديابي از راه دور اين امكان را فراهم مي‌كنند.


امنيت در سيستم حمل و نقل تنها به حمل و نقل عمومي منتهي نمي‌شود. بلكه از مهم‌ترین نگراني‌هاي صاحبان وسايل نقليه شخصي، اطمينان از امنيت وسيله نقليه آن‌ها است. سيستم‌هاي رديابي در پيشگيري از سرقت يا یافتن وسيله سرقت شده مي‌توانند كمک كنند. پلیس نيز با استفاده از
اطلاعاتي كه سيستم رديابي تعبیه شده در وسيله نقليه ارسال مي‌كنبد مي‌توانبد موقعيت را تشخيص
بدهد.


علاوه بر وسايل نقليه، سيستم‌هاي رديابي در كاربردهاي نظارت از راه دور و نظارت بر محیط زیست نيز نقش مهمي دارند. به عنوان مثال رديابي حيوانات، انسان‌ها و موقعيت‌يابي اشيا از كاربردهاي اين سيستم می‌باشد. در مثال نظارت بر انسان‌ها، اين سيستم براي افراد سالمند كه دارای بيماري‌هاي خاص چون آلزايمر هستند و احتمال گم كردن مسير براي آنها بالا است، يا براي امنيت كودكان مي‌تواند بسيار مفيد باشد. خانواده‌ها مي‌توانند از اين سيستم براي يافتن موقعيت سالمند يا كودك خود استفاده كنند. سيستمي كه در اين پروژه پياده‌سازي كرده‌ايم مي‌تواند در موارد مختلف مورد استفاده قرار بگيرد.


\section{معرفی سیستم ردیابی}
اینترنت اشیاء یک بستر ارتباطی جدید در جهت برقراری ارتباط بین اشیا هوشمند می‌باشد. معرفی این بستر موجب شــده اســت تا امکانات جدیدی برای حل مسائلی همچون تعیین مکان و ردیابی اشیا متحرک از حمله وسایل نقلیه در سطح یک شهر، منطقه یا کشور فراهم گردد.
اینترنت اشیا یک بستر ارتباطی جدید است که به سرعت در حال بدست آوردن راهکارهایی در رابطه با سناریوی ارتباط از راه دور می‌باشد و انتظار می‌رود که مبادله اطلاعات در رابطه با هر شی در شبکه‌های زنجیره‌ای منابع جهانی را آسان کند، شفافیت را افزایش دهد و کارایی‌شان را بالا ببرد. به طور گسترده اینترنت اشیا می‌تواند به عنوان ستون اصلی سیستم‌های فراگیر و فعال‌سازی محیط‌های هوشمند برای سادگی در تشخیص و شناسایی اشیا و بازیابی اطلاعات از اینترنت در هر زمان و در هر مکان به کار برده شود.


از یک دیدگاه مفهومی، اینترنت اشیا متکی بر سه اصل مرتبط با توانایی اشیا هوشمند است: ۱- قابلیت شناسایی (هر چیزی خود را شناسایی کند) ۲- قابلیت انتقال (هر چیزی دست به انتقال می‌زند) ۳- قابلیت تعامل (هر چیزی دست به تعامل می‌زند) یا در میان خودشان و یا با کاربران نهایی یا سایر نهادهای فعال در شبکه. اشیا معمولا یا به صورت منحصر به فردو یا به عنوان عضوی از یک رده شناسایی می‌شوند.


یکی از مساپل مطرح امروزی، ردیابی بی‌درنگ اشیا متحرک می‌باشد که به ردیابی بی‌درنگ موقعیت فعلی یک شی متحرک معین اشاره دارد. 
سیستم ردیابی اشیا متحرک یک راه‌حل برای بسیاری از مشکلات از جمله مساپل امنیتی است. تکنولوژی است که برای مشخص کردن موقعیت شی مورد استفاده قرار می‌گیرد.
   