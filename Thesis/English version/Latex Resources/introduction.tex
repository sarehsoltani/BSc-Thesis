\chapter{Introduction}
The concept of the Internet of Things refers to objects with a unique identity and the ability to transmit data over the network, without the need for human interaction and intervention. Its primary purpose is to make objects intelligent and provide a platform through which objects can send and receive information. The Internet of Things (IoT) refers extensively to the development of computing capabilities and network communications of objects, devices, sensors, or any other item that is not typically considered a computer.
These smart objects can collect, analyze and manage data remotely \cite{1}.

The Internet of Things is a wide range of sensors and actuators that measure and process various environmental conditions. In recent years, IoT technology has grown significantly and has met many complex needs in various fields. Due to the expansion of new technologies, the production of intelligent sensors, the growth of communication technologies, and the complexity of requirements, the Internet of Things has gained a lot of power. It has led to the expansion of intelligent systems in the environment \cite{2}.

These systems must interact with each other to have a positive impact on the environment. IoT-based technologies have different requirements compared to other technologies. Typically, these systems have less memory, power consumption, and bandwidth than other systems. Most smart systems are battery-powered and located in a remote location where they cannot be charged continuously. As a result, the power consumption and coverage range for these systems, especially those implemented at the macro level, such as smart farming, smart city and home, and tracking issues, is very important. There are several wireless communication protocols, each with its unique characteristics.

The Internet of Things is a new communication platform for communicating between intelligent objects. The introduction of this platform has provided new facilities for solving problems, such as locating and tracking moving objects from vehicle attacks at the city, region, or country level. The Internet of Things (IoT) is rapidly gaining access to telecommunications scenarios, and it is expected that the exchange of information in global resource chains will facilitate.

Widely, Internet of Things can be used as the mainstay of pervasive systems and the activation of intelligent environments for ease in identifying objects and retrieving information from the Internet.
From a conceptual point of view, the Internet of Things relies on three principles related to intelligent objects' ability:
Ability to identify, transfer and interact either among themselves or with other users. Objects are usually classified either individually or as a member of a category.

One of the most important issues today is the tracking of moving objects, which refers to the immediate tracking of a specific moving object's current position. Moving object tracking systems is a solution to many problems, including security issues. It is a technology used to determine the location of an object.

One of the important applications of IoT is tracking systems that are used in various technologies. Tracking systems were developed for the transportation industry. One of the basic needs of the owners of this industry is to check the position of vehicles. The earliest systems for locating were passive systems that stored information in memory and could only be accessed when a vehicle was available. These systems are not suitable for real-time applications because they need to provide information to the user immediately. For solving this issue, active systems were created that allow the use of in-vehicle hardware and remote tracking servers.

Today, the safety of people and vehicles has become a public concern. The tracking system has provided reliable solutions to ensure people's and vehicles' safety. It has a significant impact on optimizing the quality of monitoring and management of public transportation, vehicle movements, people (children and the elderly), or It has every other moving object. Tracking system is a technology that makes it possible to determine the exact position and track of people, vehicles, or any other moving object using various methods such as the Global Positioning System \cite{3}.

In addition to vehicles, tracking systems also play an important role in remote monitoring and environmental monitoring applications. For example, tracking animals, humans, and locating objects are some of this system's applications. In human monitoring, this system can be beneficial for the elderly who have certain diseases such as Alzheimer's and are more likely to get lost, or for the safety of children. Families can use this system to find the position of their elderly or child \cite{4}.

The system we have implemented in this project can be used in various cases. One of the applications that can be imagined for the Internet of Things is implementing a system that can determine the exact position and path of each moving object at any time. In this project, we intend to build a tracking system that can identify a moving object's exact position and path.

In this project, our communication will be one-way in that the coordinates of the moving object are continuously measured by the GPS module and sent to a server. This module is continually connected to the satellite to get coordinates. GPS data is sent to the Arduino. Finally, the GSM modem sends this information to the software server. In this project, the software servers analyze the information after receiving it, and we will not have a request from the server since our communication will be one-way. In this part of the project, web-based software will be developed to process the submitted information, store it in the database, and finally convert the stored information into a display for users. An application written using stored data displays the location on a map. In this way, the object or person can be found at the current time. In the end, the current position and direction of the person will be displayed on the map. Another application of this system is to obtain high-traffic locations by analyzing the collected information.

In the continuation of this dissertation, we talk about the general architecture of the tracking system and its components. Then, we introduce the implementation method of this system using the mentioned components. Finally, we talk about the results of the system implementation and what can be done in the future based on this project.
